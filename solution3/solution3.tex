\documentclass[10pt,a4paper]{article}
\usepackage[utf8]{inputenc}
\usepackage{amsmath}
\usepackage{amsfonts}
\usepackage{amssymb}
\usepackage{amsthm}
\usepackage{polynom} % Polynomdivision



\title{\textbf{\huge Stochastik 1
\\\large Blatt 3}}
\author{Jackie Huynh (\qquad\qquad), Daniel Speck (632 13 17)}
\date{17.04.2015}

\setlength{\topmargin}{-1.0cm}
\setlength{\textheight}{650pt}

\newcommand{\bs}{\;\backslash\;}

\begin{document}


	% Titel, Autor & Abgabedatum
	\maketitle


	% Aufgabe 1
	\section*{Aufgabe 1}
	
		Widerlege $P(A|B) = 1 - P(A | B^c)$:
		\\
		\\
		Wähle
		$\displaystyle A = \{a, b, c\}, B = \{c, d, e\}, \Omega = \{a,b,c,d,e\}$
		\\
		\\
		$\displaystyle \Rightarrow P(A \cap B) = P(\{c\}) = \frac{1}{5}$
		$\displaystyle ,\; P(A \cap B^c) = P(\{a,b\}) = \frac{2}{5}$
		\\
		\\
		Außerdem ist $\displaystyle P(B) = \frac{|B|}{|\Omega|} =\frac{3}{5}, \; P(B^c) = \frac{|\{a,b\}|}{|\Omega|} = \frac{2}{5}$
		\\
		\\
		Daraus folgt dann:
		\\
		\\
		$\displaystyle P(A|B) = \frac{P(A \cap B)}{P(B)} = \frac{\frac{1}{5}}{\frac{3}{5}} = \frac{1}{3}$
		\\
		\\
		\\
		$1 - P(A|B^c) = 1 - \displaystyle\frac{P(A \cap B^c)}{P(B^c)} = 1 - \frac{\frac{2}{5}}{\frac{2}{5}} = 1 - 1 = 0$
		\\
		\\
		\\
		Damit ergibt sich: $P(A|B) = \displaystyle\frac{1}{3} \neq 1 = P(A|B^c)$
		\begin{flushright}
			\text{$\Box$}
		\end{flushright}

		
		


\end{document}





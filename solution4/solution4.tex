\documentclass[10pt,a4paper]{article}
\usepackage[utf8]{inputenc}
\usepackage{amsmath}
\usepackage{amsfonts}
\usepackage{amssymb}
\usepackage{amsthm}
\usepackage{tabto}
\usepackage{polynom} % Polynomdivision



\title{\textbf{\huge Stochastik 1
\\\large Blatt 4}}
\author{Jackie Huynh (638 88 88), Daniel Speck (632 13 17)}
\date{04.05.2015}

\setlength{\topmargin}{-1.0cm}
\setlength{\textheight}{650pt}

\NumTabs{4}

\newcommand{\bs}{\;\backslash\;}

\begin{document}


	% Titel, Autor & Abgabedatum
	\maketitle



	\section*{Aufgabe 1}
		
		Ereignisse:
		\begin{itemize}
			\item $S$:   \tabto{2.5em} Studierender kennt die richtige Antwort
			\item $S^c$: \tabto{2.5em} Studierender kennt die richtige Antwort nicht
			\item $A$:   \tabto{2.5em} Richtige Antwort ist angekreuzt
			\item $A^c$: \tabto{2.5em} Falsche Antwort ist angekreuzt
		\end{itemize}
		$P(S) = 0.7, \quad P(A|S) = 0.7, \quad P(S^c) = 1 - 0.7 = 0.3, \quad P(A) = ?, \quad P(A^c) = ?$
		\\
		\\
		$P(A^c|S^c) = \frac{2}{3} * 0.3 = 0.2$
		\\
		\\
		(a) Mit welcher Wahrscheinlichkeit wird die richtige Antwort angekreuzt?
		\\
		\\
		\begin{displaymath}
		P(A) = P(A|S) + P(A|S^c) = \frac{P(A \cap S)}{P(S)} + \frac{P(A \cap S^c)}{P(S^c)}
		= P(A|S) * P(S) + P(A|S^c) * P(S^c)
		\end{displaymath}
		\begin{displaymath}
		= 0.7*0.7 + P(A|S^c)*0.3 = 0.7*0.7 + (1 - P(A^c|S^c)) * 0.3 = 0.49 + (1 - 0.2) * 0.3
		\end{displaymath}
		\begin{displaymath}
		= 0.49 + 0.24 = 0.73
		\end{displaymath}
		(b) $P(S|A)$: \quad Die richtige Antwort wurde angekreuzt, unter der Bedingung, dass der Student die richtige Antwort kannte (nicht geraten hat):
		\begin{displaymath}
		P(S|A) \stackrel{bayes}{=} \frac{P(A|S)*P(S)}{P(A|S)*P(S) + P(A|S^c) * P(S^c)}
		= \frac{0.7 * 0.7}{0.7 * 0.7 + 0.8 * 0.3} = \frac{0.49}{0.49+0.24} = \frac{0.49}{0.73}
		\end{displaymath}


	\newpage


	\section*{Aufgabe 2}
	
		$\Omega = \{1, ..., 6\}$,
		\tab $(\Omega, P)$ sei Laplacemaß
		\\
		\\
		$A = \{1,2,3,4\}$,
		\tab
		$B = C = \{4,5,6\}$
		\\
		\\
		(a)
		\\
		\\
		$A$ und $B$ sind nicht stochastisch unabhängig:
		\begin{displaymath}
		P(A \cap B) = P(\{4\}) = \frac{|\{4\}|}{|\Omega|}
		= \frac{1}{6}
		\neq
		\frac{1}{3}= \frac{12}{36} = \frac{4}{6} * \frac{3}{6}
		= \frac{|A|}{|\Omega|} * \frac{|B|}{|\Omega|} = P(A) * P(B)
		\end{displaymath}
		\\
		$B$ und $C$ sind nicht stochastisch unabhängig:
		\begin{displaymath}
		P(B \cap C) = \frac{|\{4,5,6\}|}{|\Omega|} = \frac{3}{6}
		\neq \frac{1}{4} = \frac{3}{6} * \frac{3}{6} = P(B) * P(C)
		\end{displaymath}
		\\
		$A$ und $C$ sind nicht stochastisch unabhängig (siehe
		nicht stochastische Unabhängigkeit von $A$ und $B$):
		\begin{displaymath}
		P(A \cap C) = P(\{4\}) = \frac{1}{6}
		\neq
		\frac{1}{3} = \frac{4}{6} * \frac{3}{6}
		= P(A) * P(C)
		\end{displaymath}
		\\
		(b)
		\\
		\\
		Ferner gilt ebenfalls $P(A \cap B \cap C) \neq P(A) * P(B) * P(C)$:
		\begin{displaymath}
		P(A \cap B \cap C)
		= P(\{4\}) = \frac{1}{6}
		\neq
		\frac{2}{27}
		= \frac{2}{3} * \frac{1}{3} * \frac{1}{3}
		= \frac{4}{6} * \frac{3}{6} * \frac{3}{6}
		= P(A) * P(B) * P(C)
		\end{displaymath}
		\\
		(c)
		\\
		\\
		Nein, die Mengen $A, B, C$ sind nicht stochastisch unabhängig, denn dafür müssten $A, B$ und $A, C$ und $B, C$ sowie $A, B, C$ stochastisch unabhängig sind, was nach (a) und (b) nicht der Fall ist.
		
		\begin{flushright}
			\text{$\Box$}
		\end{flushright}
		


\end{document}





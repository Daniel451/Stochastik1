\documentclass[10pt,a4paper]{article}
\usepackage[utf8]{inputenc}
\usepackage{amsmath}
\usepackage{amsfonts}
\usepackage{amssymb}
\usepackage{amsthm}
\usepackage{polynom} % Polynomdivision



\title{\textbf{\huge Stochastik 1
\\\large Blatt 1}}
\author{Jackie Huynh (), Daniel Speck (632 13 17)}
\date{13.04.2015}

\setlength{\topmargin}{-1.0cm}
\setlength{\textheight}{650pt}

\newcommand{\bs}{\;\backslash\;}

\begin{document}


	% Titel, Autor & Abgabedatum
	\maketitle


	% Aufgabe 1
	\section*{Aufgabe 1}
	
		(a) \quad
		$(A \bs B) \cup (A \bs C) = A \bs (B \cap C)$
		\\
		$\Leftrightarrow (A \cap B^c) \cup (A \cap C^c) = A \bs (B \cap C)$
		\\
		$\Leftrightarrow A \cap (B^c \cup C^c) = A \bs (B \cap C)$
		\\
		$\Leftrightarrow A \cap (B^c \cup C^c) = A \cap (B^c \cup C^c)$
		\\
		\\
		Da per Äquivalenzumformungen die Gleichheit beider Seiten gezeigt werden können, ist die Richtigkeit der Gleichung bestätigt.
		\\
		\\
		\\
		(b) \quad
		$(A \cup B) \bs C = A \cup (B \bs C)$
		\\
		\\
		Diese Gleichung ist inkorrekt, Beweis:
		\\
		Sei $M_1 = \{\; x \;|\; x \in A \lor x \in B, x \not\in C \;\}$ die linke Seite der Gleichung, so ist ein Element genau dann in der resultierenden Menge $M_1$ enthalten, wenn es in $A$ oder $B$, nicht aber in $C$ enthalten ist.
		\\
		Sei $M_2 = \{\; x \;|\; x \in A \lor (x \in B \land x \not\in C) \;\}$ die rechte Seite der Gleichung, so ist ein Element genau dann in der resultierenden Menge $M_2$ enthalten, wenn es in $A$ enthalten ist oder aber in $B$ und zugleich nicht in $C$.
		\\
		Das bedeutet, in der Menge $M_1$ kann kein Element aus $A \cap C$ enthalten sein, in der Menge $M_2$ sind aber alle Elemente aus $A \cap C$ enthalten.
		\\
		Damit gilt $M_1 \neq M_2$, sodass die Gleichung also inkorrekt ist.
		
		\begin{flushright}
			\text{$\Box$}
		\end{flushright}


	% Aufgabe 2
	\section*{Aufgabe 2}

		\begin{math}
			\text{(a)} \quad
			L_1 = A \cap B \cap C
			\\
			\\
			\text{(b)} \quad
			L_2 =
			(A \cap (B^c \cup C^c)) \cup
			(B \cap (A^c \cup C^c)) \cup
			(C \cap (A^c \cup B^c))
			\\
			\\
			\text{(c)} \quad
			L_3 =
			(A \cap B) \cup (A \cap C) \cup (B \cap C)
			\\
			\\
			\text{(d)} \quad
			L_4 = L_1 \cup L_2
		\end{math}

		\begin{flushright}
						\text{$\Box$}
		\end{flushright}

	\newpage

	% Aufgabe 3
	\section*{Aufgabe 3}
		
		(a) \quad
		$\Omega = \{\; (\omega_1, \omega_2, \omega_3) \;|\; \omega_1, \omega_2, \omega_3 \in \{1, 2\} \;\}$
		\\
		\\
		$\omega_i = 1 \quad \mathrel{\hat{=}} \quad$ Münze zeigt Zahl 
		\\
		$\omega_i = 2 \quad \mathrel{\hat{=}} \quad$ Münze zeigt Kopf










	\section*{Aufgabe 3}

		\begin{text}
			a.) 
			\\
			\\
			Urnenmodell: Ziehen mit Reihenfolge und ohne Zurücklegen
			\\
			\\
			$\Omega_{R,-Z} := \{ \omega \in \{ 1, ..., n \}^k \; | \; \forall i \neq j \in \{ 1, ..., k \} : \omega_i \neq \omega_j \}$
			\\
			\\
			b.)
			\\
			\\
			Herr Mayer $:= 1$, \; Frau Müller $:= 2$
			\\
			\\
			$A = \{ \omega \in \Omega_{R,-Z} \; | \; \exists i \in \{ 1, ... , 11 \} : \\(w_i = 1, w_{i+1} = 2) \vee \\(w_i = 2, w_{i+1} = 1) \vee \\(w_1 = 1, w_{12} = 2) \vee \\(w_1 = 2, w_{12} = 1) \}$
			\\
			\\
			Stellt man sich den runden Tisch bildlich vor, so sitzen Herr Mayer und Frau Müller genau dann hintereinander (also zusammen), wenn $w_i = 1, w_{i+1} = 2$ oder $w_i = 2, w_{i+1} = 1$ gilt. Ebenso sitzen sie natürlich zusammen, wenn $w_1 = 1, w_{12} = 2)$ oder $(w_1 = 2, w_{12} = 1)$ gilt, denn dann sitzt einer von beiden auf Platz 1, der andere auf Platz 12, was bei einem runden Tisch mit 12 Plätzen ebenso bedeutet, dass sie nebeneinander sitzen.
			\\
			\\
			c.)
			\\
			\\
			$|A| = ( \sum_{i=1}^{11}$ 
			$( \, |\{ \omega : \omega_i = 1, \omega_{i+1} = 2\}| $
			$+$
			$|\{ \omega : \omega_i = 2, \omega_{i+1} = 1\}| \,) )$
			$+$
			$|\{ \omega : \omega_1 = 1, \omega_{12} = 2\}|$
			$+$
			$|\{ \omega : \omega_1 = 2, \omega_{12} = 1\}| \;$
			\\
			\\
			$ = 11 * ( 10! + 10! ) + 10! + 10!$
			\\
			\\
			$ = 11 * ( 2 * 10! ) + 2 * 10!$
			\\
			\\
			$ = 2 * 11! + 2 * 10! = 2 * ( 11! + 10! )$
			\\
			\\
			$ \Rightarrow P(A) = \frac{2 * (11! + 10!)}{12!} = \frac{2}{11}$
		\end{text}

\end{document}





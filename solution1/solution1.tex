\documentclass[10pt,a4paper]{article}
\usepackage[utf8]{inputenc}
\usepackage{amsmath}
\usepackage{amsfonts}
\usepackage{amssymb}
\usepackage{amsthm}
\usepackage{polynom} % Polynomdivision



\title{\textbf{\huge Stochastik 1
\\\large Blatt 1}}
\author{Franziska Kissel (643 35 57), Daniel Speck (632 13 17)}
\date{13.04.2014}

\setlength{\topmargin}{-1.0cm}
\setlength{\textheight}{650pt}


\begin{document}


	% Titel, Autor & Abgabedatum
	\maketitle




	% Aufgabe 1
	\section*{Aufgabe 1}


		\begin{math}
			\text{a.)}
			\\
			\\
			A^c \cap B^c \cap C^c = ( A \cup B \cup C )^c
			\\
			\\
			\text{b.)}
			\\
			\\
			( A \cap B ) \cup ( A \cap C ) \cup ( B \cap C )
		\end{math}

		\begin{flushright}
						\text{$\Box$}
		\end{flushright}


	\section*{Aufgabe 2}

		\begin{text}
			I. Zeige $P(A \cap B) \le \frac{1}{3}$:
			\\
			\\
			Die Schnittmenge zweier Mengen $A, B$ kann höchstens so mächtig sein, wie die weniger mächtige der beiden Mengen, da für die Schnittmenge formal gilt:
			\\
			\\
			$A \cap B  = \{ x : x \in A \wedge x \in B\}$
			\\
			\\
			Da $P(B) = \frac{1}{3} < \frac{3}{4} = P(A)$ gegeben ist, muss der Menge $B$ offensichtlich mindestens ein Element fehlen, welches für die höhere Wahrscheinlichkeit der Menge $A$ sorgt (ansonsten wäre $P(B) \ge P(A)$, wenn $B$ alle Elemente aus $A$ enthalten würde).
			\\
			Damit wäre gezeigt, das für $P(A \cap B)$ maximal $P(A \cap B) = P(B)$ gelten kann, da $B$ nicht alle Elemente aus A enthält und somit der Schnitt maximal alle aus B enthalten kann.
			\\
			Hiermit haben wir dann die obere Grenze $P(A \cap B) \le \frac{1}{3}$ gefunden.
			\\
			\\
			II. Zeige $\frac{1}{12} \le P(A \cap B)$:
			\\
			\\
			$P(A) + P(B) = \frac{13}{12} \quad \Rightarrow \quad P(A \cap B) \neq \emptyset$, da die Summe aller Wahrscheinlichkeiten des Wahrscheinlichkeitenraumes 1 ergeben muss (man hat also offensichtlich bereits bei der Summierung aller Wahrscheinlichkeiten der Ereignisse A und B mindestens 1 Element doppelt gezählt bei $P(A) + P(B)$, woraus sich ergibt, dass mindestens 1 Element in beiden Mengen enthalten sein muss. Wäre das nicht der Fall, so müsste ansonsten $P(A) + P(B) \le 1$ gelten).
			\\
			\\
			Schaut man sich nun die Summe von $P(A) + P(B) = \frac{9}{12} + \frac{4}{12} = \frac{13}{12}$ an, fällt außerdem direkt auf, dass die Summe genau $\frac{1}{12}$ über $1$ liegt. Daraus folgt, dass eine Teilmenge $Y$ mit $A \cap B = Y$ existiert, für die gilt, dass $P(Y) \ge \frac{1}{12}$.
			\\
			Damit wissen wir, dass die Schnittmenge von $A, B$ nicht leer ist und dies impliziert auch, dass das Ergebnis der Schnittmenge, die Menge $Y$, mindestens ein Element haben muss.
			\\
			Da für die Wahrscheinlichkeit von $Y$ gilt, dass sie mindestens $\frac{1}{12}$ ist, haben wir unsere Untere Grenze für die Wahrscheinlichkeit der Schnittmenge gefunden. Es gilt also $\frac{1}{12} \le P(A \cap B)$
			\\
			\\
			Aus I und II folgt dann $\frac{1}{12} \le P(A \cap B) \le \frac{1}{3}$
			\begin{flushright}
						q.e.d.
			\end{flushright}
		\end{text}



	\section*{Aufgabe 3}

		\begin{text}
			a.) 
			\\
			\\
			Urnenmodell: Ziehen mit Reihenfolge und ohne Zurücklegen
			\\
			\\
			$\Omega_{R,-Z} := \{ \omega \in \{ 1, ..., n \}^k \; | \; \forall i \neq j \in \{ 1, ..., k \} : \omega_i \neq \omega_j \}$
			\\
			\\
			b.)
			\\
			\\
			Herr Mayer $:= 1$, \; Frau Müller $:= 2$
			\\
			\\
			$A = \{ \omega \in \Omega_{R,-Z} \; | \; \exists i \in \{ 1, ... , 11 \} : \\(w_i = 1, w_{i+1} = 2) \vee \\(w_i = 2, w_{i+1} = 1) \vee \\(w_1 = 1, w_{12} = 2) \vee \\(w_1 = 2, w_{12} = 1) \}$
			\\
			\\
			Stellt man sich den runden Tisch bildlich vor, so sitzen Herr Mayer und Frau Müller genau dann hintereinander (also zusammen), wenn $w_i = 1, w_{i+1} = 2$ oder $w_i = 2, w_{i+1} = 1$ gilt. Ebenso sitzen sie natürlich zusammen, wenn $w_1 = 1, w_{12} = 2)$ oder $(w_1 = 2, w_{12} = 1)$ gilt, denn dann sitzt einer von beiden auf Platz 1, der andere auf Platz 12, was bei einem runden Tisch mit 12 Plätzen ebenso bedeutet, dass sie nebeneinander sitzen.
			\\
			\\
			c.)
			\\
			\\
			$|A| = ( \sum_{i=1}^{11}$ 
			$( \, |\{ \omega : \omega_i = 1, \omega_{i+1} = 2\}| $
			$+$
			$|\{ \omega : \omega_i = 2, \omega_{i+1} = 1\}| \,) )$
			$+$
			$|\{ \omega : \omega_1 = 1, \omega_{12} = 2\}|$
			$+$
			$|\{ \omega : \omega_1 = 2, \omega_{12} = 1\}| \;$
			\\
			\\
			$ = 11 * ( 10! + 10! ) + 10! + 10!$
			\\
			\\
			$ = 11 * ( 2 * 10! ) + 2 * 10!$
			\\
			\\
			$ = 2 * 11! + 2 * 10! = 2 * ( 11! + 10! )$
			\\
			\\
			$ \Rightarrow P(A) = \frac{2 * (11! + 10!)}{12!} = \frac{2}{11}$
		\end{text}

\end{document}





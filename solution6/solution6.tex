\documentclass[10pt,a4paper]{article}
\usepackage[utf8]{inputenc}
\usepackage{amsmath}
\usepackage{amsfonts}
\usepackage{amssymb}
\usepackage{amsthm}
\usepackage{tabto}
\usepackage{polynom} % Polynomdivision



\title{\textbf{\huge Stochastik 1
\\\large Blatt 6}}
\author{Jackie Huynh (638 88 88), Daniel Speck (632 13 17)}
\date{18.05.2015}

\setlength{\topmargin}{-1.0cm}
\setlength{\textheight}{650pt}

\NumTabs{4}

\newcommand{\bs}{\;\backslash\;}

\begin{document}


	% Titel, Autor & Abgabedatum
	\maketitle



	\section*{Aufgabe 1}
	
		
		
	\section*{Aufgabe 2}
	
	\newpage
	
	\section*{Aufgabe 3}
	
		(a) \quad Da die Personen unabhängig voneinander ausgewählt wurden und alle dieselbe Wahrscheinlichkeit haben, die Blutgruppe AB negativ zu haben, können wir als zugrundeliegende Verteilung folgendes annehmen:
		\\
		\\
		$X$ sei die Verteilung und $X$ ist binomialverteilt mit $Bin_{(50, \; 0.01)}$
		\\
		\\
		Gesucht ist nun $P(X \ge 1) = 1 - P(X=0) = 1 - \binom{50}{0} * 0.01^0 * (1-0.01)^{50-0}$
		\\
		\\
		$= 1 - 1 * 1 * (0.99)^{50} \approx 1 - 0.61 = 0.39$
		\\
		\\
		Die Wahrscheinlichkeit, dass eine Person dieser 50 die Blutgruppe AB negativ hat, beträgt also ca. 39\%.
		\\
		\\
		Eine approximative Lösung ist mittels der Poissonverteilung möglich. 
		\\
		\\
		Allgemein gilt für eine derartige Verteilung:
		\begin{equation}
		\label{poisson_allgemein}
			P_{\lambda}(\{k\})
			= e^{-\lambda} * \frac{\lambda^{k}}{k!}
		\end{equation}
		In dieser Aufgabe ist $\lambda$ mit $\lambda = n*p = 50 * 0.01 = 0.5$ gegeben.
		\\
		Damit lässt sich die Wahrscheinlichkeit wie folgt approximieren:
		\begin{equation}
			\label{poisson_komplement}
			P_{\lambda}(\{k \ge 1\})
			= \sum\limits_{i=1}^{k} e^{-\lambda} * \frac{\lambda^{i}}{i!}
			= 1 - P_{\lambda}(\{k=0\})
			= 1 - e^{-\lambda} * \frac{\lambda^{k}}{k!}
		\end{equation}
		\\
		Denn die Gesamtwahrscheinlichkeit ergibt $1$ und $P_{\lambda}(\{k = 0\})$ ist das Komplement von $P_{\lambda}(\{k \ge 1\})$, sodass sich gerade $P_{\lambda}(\{k \ge 1\}) = 1 - P_{\lambda}(\{k = 0\})$ ergibt.
		\\
		\\
		Daraus ergibt sich:
		\begin{equation*}
			P_{\lambda}(\{k \ge 1\})
			= 1 - P_{\lambda}(\{k = 0\})
			= 1 - e^{-\lambda}
			* \underbrace{\frac{\lambda^{0}}{0!}}_{\substack{=1}} 
			= 1 - e^{-\lambda}
			= 1 - e^{-0.5}
			\approx 0.39
		\end{equation*}
		\\
		Damit ergibt sich aus der Poissonverteilung gerundet ebenfalls eine Wahrscheinlichkeit von 39\% (approximiert).
		\\
		\\
		(b) \quad Da eine geeignete Approximation reicht, kann hier die Poissonverteilung benutzt werden, da $n \ge 50$ aufgrund der 50 Personen erfüllt ist und $p \le 0.05$ gilt, da die Wahrscheinlichkeit hier $p = 0.01$ ist.
		\\
		\\
		Für mindestens eine Person gilt:
		\begin{equation}
			\label{3b_verteilung}
			P_{\lambda}(\{k \ge 1\})
			\overset{(\ref{poisson_komplement})}{=}
			1 - e^{-\lambda} * \underbrace{\frac{\lambda^{0}}{0!}}_{\substack{=1}}
			\overset{(\lambda = n*p)}{=}
			1 - e^{-n*p} 
		\end{equation}
		Nun ist $n$ gesucht für $p = 0.01$ und einer Gesamtwahrscheinlichkeit von mindestens 50\%, dass das Ereignis zutrifft:
		\begin{gather}
			\nonumber
			0.5 \overset{(\ref{3b_verteilung})}{\le} 1 - e^{-n*p}
			\;\Leftrightarrow\; 0.5 + e^{-n*p} \le 1
			\;\Leftrightarrow\; e^{-n*p} \le 0.5
			\\
			\nonumber
			\Leftrightarrow\;   ln(e^{-n*p}) \le ln(0.5)
			\;\Leftrightarrow\; -n*p \le ln(0.5)
			\\
			\label{3b_ergebnis}
			\Leftrightarrow\;   n \ge -\frac{ln(0.5)}{p}
			= -\frac{ln(0.5)}{0.01} \approx 69.31
		\end{gather}
		Rundet man das Ergebnis aus (\ref{3b_ergebnis}) ab, so kommt man zu dem Schluss, dass man mindestens eine Gruppe von 70 voneinander unabhängig ausgewählten Personen benötigt, um mit einer Wahrscheinlichkeit von mindestens 50\% davon ausgehen kann, dass mindestens einer aus dieser Gruppe die Blutgruppe AB negativ hat.
		


		\begin{flushright}
			\text{$\Box$}
		\end{flushright}
		


\end{document}




